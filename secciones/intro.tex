
\chapter{Introducción}
\label{ch:intro}

\section{Motivación y Necesidad}
\label{sec:motivacionyobjetivos}

Las técnicas de Monitorización No Intrusiva de Carga (en inglés NILM o Non Intrusive Load Monitoring) extraen el consumo de electrodomésticos a partir de una curva de consumo agregada. \autocite[1]{NILMreview}. Dicho de otra forma. El monitoreo de carga no intrusivo es un sistema que infiere los consumos de diferentes dispositivos sobre una curva de consumo en la toma de corriente principal. Por ejemplo en la toma de corriente de una casa.

En Europa, un tercio de la energía eléctrica se ha generado mediante energías no renovables(12\% o 333TWh carbón, 17\% o 569TWh gas natural)\autocite{energyeurope} \autocite{ember2024european}. Aún siendo un descenso récord, para poder continuar reduciendo el consumo de energías no renovables, debemos desarrollar herramientas que nos permitan hacer un uso más inteligente de la generación de energía. Herramientas como la monitorización de carga no intrusiva pueden ayudarnos a detectar patrones de uso\autocite[11]{nilmstateoftheart}, para entonces poder desarrollar políticas que permitan optimizar la generación de energía. 

\todo[inline]{estimar el ahorro en emisiones si se optimizase el consumo sólo un 2\%}



Realizar un monitoreo de carga no intrusivo es un problema complejo. Actualmente el área de soluciones NILM se encuentra en la fase de investigación y desarrollo. Por lo tanto. Mi objetivo para este trabajo es entrenar y analizar el rendimiento de una inteligencia artificial especializada en la clasificación de imágenes, transformando los datos de entrenamiento de series temporales a imágenes. Para más detalle, dirigirse al \textbf{\autoref{ch:desarrollo}}. Quiero ver cómo modificar el enfoque al problema afecta al rendimiento de la técnica en comparación con otras técnicas del estado del arte. 

Siguiendo este objetivo y con el asesoramiento de Esteban García Cuesta, se decidió en hacer uso de datasets clasificados para un entrenamiento supervisado; además de una arquitectura concreta, compacta y computacionalmente baja en costes de entrenamiento. 

\section{title}

