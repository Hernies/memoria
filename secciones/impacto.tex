\chapter{Resultados y Análisis de Impacto}

\section{Resultados}

\subsection{Costes computacionales}
\todo[inline]{Costes computacionales del modelo, tiempo de entrenamiento, en comparación con otros modelos.}

\subsection{Evaluación del modelo}
Para evaluar el modelo, se hace uso de las mismas métricas definidas por Ang Gao en \autocite{GAO2023109443}: accuracy, F1 score, and silhouette score\footnote{modificadas para el análisis multi-etiqueta}.
\todo[inline]{Precisión del modelo, comparación con otros modelos, disclaimer! no son comparaciones muy útiles ya que habría que adaptar el dataset para utilizar los datos de la misma forma y analizar.}


\todo[inline]{Resultados del modelo: el transfer learning se considera exitoso para la regresión temporal, no para la clasificación. comparación con otros modelos.}

\section{Impacto}
\subsection{Personal}
Ha sido una experiencia académicamente enriquecedora. Ya que cuando se empezó este trabajo, no se tenía prácticamente ningún conocimiento sobre la inteligencia artificial a un nivel que personalmente se considerase razonable. Se termina este trabajo habiendo disfrutado de profundizar en el área de la inteligencia artificial, más allá de lo estrictamente necesario para el desarrollo de este trabajo. 
\subsection{Social}
%%\todo[inline]{Mayor control de gasto, menos privacidad. Analisis de patrones de gastos y por tanto hábitos personaless/de la vivienda}
Dados los resultados del modelo, el impacto de construir un sistema haciendo uso del modelo entrenado tendrían alguna serie de problemas con la detección de clases de dispositivos. Pero obviando esto; o clasificando manualmente los dispositivos presentes dentro de la vivienda. Podría tener un uso muy ventajoso de cara a un usuario. 
A cambio, podría presentar una serie de problemas de privacidad, si no se manejara de forma segura la seguridad del sistema. 
\subsection{Empresarial}
%%\todo[inline]{industria: mayor control y más barato. (explicar pq) del gasto energético}
Presenta las mismas ventajas e inconvenientes que en el ámbito social. 
\subsection{Medioambiental}
%%\todo[inline]{pro: aprovechar el consumo/optimizar los patrones de gasto teniendo mayor granularidad y un modelo inferente como el que se ha desarrollado permite construir en un futuro herramientas predictivas de gastos que por tanto pueden aportar info valiosa para modelos más generales (usando estos y otros datos como los metereológicos, mercantiles, mercados, etc) para estimar la demanda esperada y el volumen de energía a generar.}
El potencial de un modelo como este puede beneficiar la implementación de sistemas de optimización de generación y consumo de energía, que podría potenciar el establecimiento de más comunidades solares en España. Además, establecer este tipo de sistemas da pie a la posibilidad de minar datos para poder entrenar modelos más generales que incorporen los datos de consumo de usuarios y otros, como datos metereológicos, precio de la electricidad, etc. Para poder así estimar y gestionar la oferta y demanda. 

Se recomienda analizar también el potencial impacto respecto a los Objetivos de Desarrollo Sostenible (ODS), de la Agenda 2030, que sean relevantes para el trabajo realizado (\href{https://www.un.org/sustainabledevelopment/es/objetivos-de-desarrollo-sostenible/}{ver enlace})
