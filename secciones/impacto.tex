\chapter{Análisis de impacto}

\subsection{Personal}
\todo[inline]{Mayor control de gasto, menos privacidad. Analisis de patrones de gastos y por tanto hábitos personaless/de la vivienda}

\subsection{Empresarial}
\todo[inline]{industria: mayor control y más barato. (explicar pq) del gasto energético }
\subsection{Económico}
\todo[inline]{mejora de la gestión energética lleva a mayor eficiencia, optimización y a mejor aprovechamiento de las energías renovables}
\subsection{Medioambiental}
\todo[inline]{pro: aprovechar el consumo/optimizar los patrones de gasto teniendo mayor granularida y un modelo inferente como el que se ha desarrollado permite construir en un futuro herramientas predictivas de gastos que por tanto pueden aportar info valiosa para modelos más generales (usando estos y otros datos como los metereológicos, mercantiles, mercados, etc) para estimar la demanda esperada y el volumen de energía a generar.}



Se recomienda analizar también el potencial impacto respecto a los Objetivos de Desarrollo Sostenible (ODS), de la Agenda 2030, que sean relevantes para el trabajo realizado (\href{https://www.un.org/sustainabledevelopment/es/objetivos-de-desarrollo-sostenible/}{ver enlace})
