\chapter{Desarrollo}
\label{ch:desarrollo}
\todo[inline]{TODOS LOS DIAGRAMAS SE HARAN CON MIRO, EXPORTADOS A SVG E INTEGRADOS}

\section{Planificación}
\label{sec:planificacion}
Se realizó dos tandas de organización y planificación de tareas. La primera al principio del semestre y la segunda a mediados de abril. 
Las dificultades en la planificación fueron una combinación de motivos externos y una perspectiva algo ingenua del tiempo a invertir y el ritmo de progreso.
Debido a la falta de conocimientos, se ha dedicado un tiempo considerable (unas 60-80 horas) a realizar investigación y estudio, para poder comprender las técnicas  en el estado del arte y en lo 

Los principales problemas fueron: la dificultad para programar un codificador en C++ y OpenCL, debido a la curva de dificultad que representa aprender un lenguaje para una nueva arquitectura y los retos que presenta la gran cantidad de datos usada para el entrenamiento de inteligencia artificial. 


\section{REFIT Dataset}
\subsection{Pros y contras del dataset}
\subsection{Características principales}
\subsection{Recogida de datos}
\todo[inline]{Hablar de REFIT, sus características, los datos y los metadatos. Especial hincapié en cómo se recogieron}

\section{Parseado de REFIT a MYSQL}
\subsection{Motivación}
\subsection{Inconvenientes}
\subsection{Solución}
\todo[inline]{Hablar de porqué lo parseamos a mysql, los inconvenientes (zona horaria e interrupciones en el parseo) y el diseño realizado para resolver estos problemas, además de los programas desarrollados (el primero que se rompía y el segundo que es traversecsv.cpp) }

\section{Codificación GAF}
\subsection{Justificación del uso de OpenCL}
\subsection{Inconvenientes}
\subsection{Solución}
\todo[inline]{Hablar de porqué se decidió usar opencl, porqué se retrasó tanto esta parte del proyecto el diseño que era muy bueno pero muy complicado de implementar}


\section{Entrenamiento}
\subsection{ZLUDA y Darknet Framework}
\subsection{Monitorización}
\subsection{Inconvenientes}
\todo[inline]{Apartado pendiente}
