\chapter{Desarrollo}
\label{ch:desarrollo}
Capítulo dedicado a describir el desarrollo del trabajo realizado. De acuerdo con el tutor, este capítulo puede tener distintas estructuras, e incluso pueden existir \textbf{varios capítulos}.

Por ejemplo, para un trabajo de desarrollo clásico, este capítulo de ``Desarrollo'' podría convertirse en tres capítulos:

\begin{itemize}
\item Requisitos (con la especificación de requisitos, ya sea clásico o ``agile'')
\item Análisis y diseño (con  modelo de datos, arquitectura, diseño de experiencia de usuario e interfaz de usuario, etc.)
\item Implementación (con detalles sobre la solución/implementación realizada).
\end{itemize}

Todos los capítulos deben empezar en una página nueva (esta plantilla ya lo hace automáticamente).

Los apartados dentro de los capítulos se numeran de forma jerárquica, pero siempre deben estar alineados al margen izquierdo. Ejemplo:

\section{Apartado 1 de capítulo 2}

\subsection{Sección 1 de apartado 1 de capítulo 2}

\subsubsection{Sub sección 1}

\paragraph{Párrafo 1} Con cierto contenido

\paragraph{Párrafo 2} Con más contenido

\subsubsection{Sub sección 2}

\paragraph{Párrafo 1} Con cierto contenido

\paragraph{Párrafo 2} Con más contenido

\subsection{Sección 2 de apartado 1 de capítulo 2}

\section{Apartado 2 de capítulo 2}

\section{Apartado 3 de capítulo 2}
