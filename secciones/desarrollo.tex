\chapter{Desarrollo}
\label{ch:desarrollo}
\todo[inline]{TODOS LOS DIAGRAMAS SE HARAN CON MIRO, EXPORTADOS A SVG E INTEGRADOS}

\subsection{REFIT Dataset}
\todo[inline]{Hablar de REFIT, sus características, los datos y los metadatos. Especial hincapié en cómo se recogieron}
\subsubsection{Pros y contras del dataset}
\subsubsection{Características principales}
\subsubsection{Recogida de datos}

\subsection{Parseado de REFIT a MYSQL}
\todo[inline]{Hablar de porqué lo parseamos a mysql, los inconvenientes (zona horaria e interrupciones en el parseo) y el diseño realizado para resolver estos problemas, además de los programas desarrollados (el primero que se rompía y el segundo que es traversecsv.cpp) }
\subsubsection{Motivación}
\subsubsection{Inconvenientes}
\subsubsection{Solución}

\subsection{Codificación GAF}
\todo[inline]{HAblar de porqué se decidió usar opencl, porqué se retrasó tanto esta parte del proyecto el diseño que era muy bueno pero muy complicado de implementar}
\subsubsection{Justificación del uso de OpenCL}
\subsubsection{Inconvenientes}
\subsubsection{Solución}


\subsection{Entrenamiento}
\subsubsection{ZLUDA y Darknet Framework}
\subsubsection{Monitorización}
\subsubsection{Inconvenientes}

